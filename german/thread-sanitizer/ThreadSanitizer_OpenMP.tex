
\documentclass[xcolor=dvipsnames,hyperref={pdfpagelabels=false}]{beamer}

\usepackage{courier}                 % "courier" als Schriftart einbinden
\usepackage{icomma}                  % Deutsche Kommas bei Zahlen
\usepackage{xcolor}

\usepackage[utf8]{inputenc}          % UTF-8 codierte Dateien
\usepackage[english, ngerman]{babel} % Setze die Sprache(n) für Rechtschreibung, etc.
                                     % Die letzte Sprache ist die Beginnende.
\usepackage{lmodern}                 % Bessere Schriftart


%\usetheme{Berlin}

%Damit wir Quellcode nutzen können.
\usepackage{listings}
\lstset{numbers=left,
	numberstyle=\tiny,
	numbersep=5pt,
	breaklines=true,
	showstringspaces=false,
	frame=l,
	tabsize=4,
	xleftmargin=15pt,
	xrightmargin=15pt,
	basicstyle=\ttfamily\scriptsize,
	stepnumber=1,
	keywordstyle=\color{blue},          % keyword style
	commentstyle=\color{dkgreen},       % comment style
	stringstyle=\color{BurntOrange}     % string literal style
}
%Sprache Festelegen
\lstset{language=C}

\useoutertheme[subsection=true]{smoothbars}

\title{ThreadSanitizer}   
\author{Andre Meyering} 
\date{\today}

\begin{document}


\begin{frame}
\titlepage
\end{frame} 

\begin{frame}
\frametitle{Inhaltsverzeichnis}
\tableofcontents
\end{frame} 


\section{Einführung} 
\begin{frame}
	\frametitle{Aktuelle Situation} 
    \begin{itemize}
    	\item parallele Ausführung wird immer wichtiger
    	\item Code wird immer paralleler\pause 
    	\item immer mehr Abstraktionen
    	\item Code wird komplexer
    	\item Race-Conditions schwer zu entdecken
    \end{itemize}
\end{frame}

\begin{frame} 
	\frametitle{Wie Race-Conditions finden?} 
	\begin{itemize}
		\item viel Nachdenken\pause 
		\item viel Kopfzerbrechen\pause 
		\item viele Tränen
	\end{itemize}
\end{frame}

\begin{frame} 
	\frametitle{Lösung - Thead Sanitizer}
	\begin{quotation}
		ThreadSanitizer is a tool that detects data races. It consists of a compiler instrumentation module and a run-time library.
	\end{quotation}

	\begin{itemize}
		\item von Google als Teil von LLVM entwickelt
		\item \footnotesize{\url{https://clang.llvm.org/docs/ThreadSanitizer.html}}
	\end{itemize}
\end{frame}


\section{Funktionsweise} 
\begin{frame}
	\frametitle{Anforderungen}
	\begin{itemize}
		\item Linux, NetBSD, FreeBSD | 64\,Bit\pause 
		\item ab GCC 4.8 (2012)\footnote{\tiny{\url{https://www.phoronix.com/scan.php?page=news_item&px=MTIzOTU}}}
		\item ab Clang \~{}3.5\pause 
		\item Erfordert die ThreadSanitizer Library
		\item \texttt{sudo apt install libtsan0}
		\item benötigt Position Independent Code (\texttt{-fPIE})
	\end{itemize} 
\end{frame}

\begin{frame}
\frametitle{Funktionsweise}
	\begin{itemize}
		\item fügt eigene Instruktionen in das Binary ein\medskip\pause 
		\item \texttt{export CFLAGS="{}-g -O2 -fsanitize=thread -fno-omit-frame-pointer"}
			\begin{itemize}
				\item Debug Informationen
				\item Optimierung für bessere Performance
				\item Frame Pointer für besseren Stacktrace
			\end{itemize}\medskip\pause 
		\item Error zur Runtime falls Race-Condition entdeckt wurde
	\end{itemize}
\end{frame}

\section{Beispiele} 
\begin{frame}[fragile]
	\frametitle{pthread | Ausgabe}
	\lstinputlisting{frame-examples/pthread_error.c}
	\smallskip
	\texttt{gcc  -g  -O2  -fsanitize=thread}\\
		\texttt{~~~-fno-omit-frame-pointer}\\
		\texttt{~~~-o pthread\_error  pthread\_error.c}
	\vspace*{1cm}
\end{frame}

\begin{frame}[fragile]
	\frametitle{pthread}
	\lstinputlisting[language={},basicstyle=\tiny]{frame-examples/pthread_error.err}
	
\end{frame}


\begin{frame}[fragile]
	\frametitle{OpenMP}
	\lstinputlisting{./frame-examples/openmp_no_error.c}
	\bigskip
	Wo ist die Race-Condition?\\\pause
	\begin{itemize}
		 \item es gibt keine\\\pause
		 \item dennoch Fehlermeldung
	\end{itemize}
\end{frame}

\section{OpenMP}
\begin{frame}
	\frametitle{OpenMP - ThreadSanitizer}
	\begin{itemize}
		\item OpenMP hat TSAN Unterstützung
		\item \url{https://xrunhprof.wordpress.com/2018/08/27/tsan-with-openmp/}
		\item CMake Flag: \texttt{-DLIBOMP\_TSAN\_SUPPORT=ON}
	\end{itemize}

	\begin{block}{Beispiel}
		\medskip	
		\texttt{clang -fno-omit-frame-pointer -g}\\
		\texttt{~~~-isystem/path/to/omp/include -L/path/to/omp/lib}\\
		\texttt{~~~-fopenmp -fsanitize=thread file.c -o out}
	\end{block}
\end{frame}

\begin{frame}
\frametitle{Beispiel}
	\lstinputlisting{./frame-examples/openmp_error.c}
	\medskip
	Wer findet die Race-Condition?
	\vspace*{1cm}
\end{frame}

\begin{frame}
	\frametitle{Fehlermeldung}
	\lstinputlisting[language={},basicstyle=\tiny]{./frame-examples/openmp_error.err}
	\vspace*{1cm}
\end{frame}

\begin{frame}
\frametitle{Achtung}
	\begin{block}{Normales OpenMP}
		\begin{itemize}
			\item ThreadSanitizer meckert bei jedem Durchlauf
			\item false-positives
		\end{itemize}
	\end{block}

	\begin{block}{ThreadSanitizer OpenMP}
		\begin{itemize}
			\item ThreadSanitizer meckert beim pthread Beispiel gar nicht
			\item keine false-positives
			\item meckert beim OpenMP Beispiel
			\item viele false-negatives (läuft nicht immer in race-condition)
		\end{itemize}
	\end{block}
\end{frame}


\section{Fazit / Ausblick}
\begin{frame}
	\frametitle{Fazit}
	\begin{block}{ThreadSanitizer}
		\begin{itemize}
			\item sehr nützlich um Race-Conditions zu finden
			\item verlangsamt Code um das 10 bis 15 fache
			\item ein Muss für neue Projekte
		\end{itemize}
	\end{block}

	\begin{block}{OpenMP + ThreadSanitizer}
		\begin{itemize}
			\item nützlich bei OpenMP Projekten
			\item keine false-positives bei \texttt{\#pragma omp parallel}
			\item meldet nicht immer Race-Conditions \newline(es muss eine stattfinden)
		\end{itemize}
	\end{block}

%\begin{exampleblock}{Blocktitel}
%Blocktext 
%\end{exampleblock}

%\begin{alertblock}{Blocktitel}
%Blocktext 
%\end{alertblock}

\end{frame}


\begin{frame}
	\frametitle{Ausblick}
	\begin{block}{Weitere Sanitizer}
		\medskip	
		\begin{itemize}
			\item AddressSanitizer
				\begin{itemize}
					\item use-after-free error
					\item LeakSanitizer
				\end{itemize}
				\medskip
			\item MemorySanitizer
				\begin{itemize}
					\item unitialized memory read
				\end{itemize}
				\medskip
			\item UndefinedBehaviorSanitizer
				\begin{itemize}
					\item signed integer overflow
					\item integer divide by zero
				\end{itemize}
			\end{itemize}
		\end{block}	
\end{frame}

\begin{frame}
	\frametitle{Danke}
	\begin{itemize}
		\item Präsentation und Projekt unter:\newline \footnotesize{\url{https://github.com/bugwelle/bugwelle-lightning-talks}}
	\end{itemize}
	\bigskip
	\bigskip
	\begin{center}
		\huge Fragen?
	\end{center}
\end{frame}

\end{document}
